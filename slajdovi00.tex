\documentclass[utf8,compress,aspectratio=169]{beamer}
\usepackage{irbookslide}
\usepackage{irilmenau2}
\usepackage{url}
\usepackage{fontspec} % zahteva paket euenc
\usepackage{xunicode}
\usepackage{xltxtra}
\usepackage{polyglossia}
\usepackage{minted}
\usepackage{xcolor,colortbl}
\usepackage{textcomp}
\usepackage{unicode-math}

\title{Informacije o predmetu}
\subtitle{\tiny{Slajdovi za predmet Osnove programiranja}}
\subject{Osnove programiranja}
\institute{Katedra za informatiku, Fakultet tehničkih nauka, Novi Sad}
\date{2022.}

\begin{document}

\frame{\titlepage}

\frame{
  \frametitle{Cilj}
  \begin{itemize}
    \item poznavanje principa i tehnika pisanja proceduralnih programa na jeziku Python
  \end{itemize}
}

\frame{
  \frametitle{Sadržaj predmeta}
  \begin{itemize}
    \item pojam programa, izvršavanja, kompajlera, interpretera
    \item proces pisanja programa
    \item promenljive i tipovi podataka, unos i ispis
    \item rukovanje brojevima
    \item rukovanje stringovima, nizovima i fajlovima
    \item pisanje funkcija
    \item grananje u programu
    \item programske petlje
    \item logički izrazi
    \item ...
  \end{itemize}
}

\frame{
  \frametitle{Organizacija nastave}
  \begin{itemize}
    \item 3 časa predavanja -- svi zajedno
    \item 3 časa vežbi -- po grupama
  \end{itemize}
}

\frame{
  \frametitle{Nastavnici}
  \begin{itemize}
    \item predavanja: Branko Milosavljević \\
      \texttt{mbranko@uns.ac.rs}
    \item vežbe: Lazar Nikolić i Nataša Rajtarov \\
      \texttt{lazar.nikolic@uns.ac.rs}, \texttt{natasarajtarov@uns.ac.rs}\\ \ \\
    \item konsultacije putem emaila
    \item ili uživo u unapred dogovorenom terminu
  \end{itemize}
}

\frame{
  \frametitle{Literatura}
  \begin{itemize}
    \item John Zelle. \textit{Python Programming: An Introduction to Computer Science}, 2nd edition.
      Franklin, Beedle \& Associates, 2010. ISBN 1590282418
  \end{itemize}
  \begin{center}
    \includegraphics[width=4cm]{pic00}
  \end{center}
}

\frame{
  \frametitle{Nastavni materijal}
  \begin{itemize}
    \item nastavni materijal:
    \begin{itemize}
      \item Canvas
    \end{itemize}
    \item slajdovi sa predavanja
    \item zadaci sa vežbi
    \item domaći zadaci
  \end{itemize}
}

\frame{
  \frametitle{Polaganje ispita}
  \begin{itemize}
    \item u toku semestra -- domaći zadatak
    \item na kraju semestra -- odbraniti projekat
    \item projekat se usmeno demonstrira i obrazlaže pred nastavnikom
    \item tom prilikom se dobijaju i teorijska i praktična pitanja
  \end{itemize}
}
\end{document}